% Created 2016-07-26 Tue 13:51
\documentclass[bigger]{beamer}
\usepackage[utf8]{inputenc}
\usepackage[T1]{fontenc}
\usepackage{fixltx2e}
\usepackage{graphicx}
\usepackage{grffile}
\usepackage{longtable}
\usepackage{wrapfig}
\usepackage{rotating}
\usepackage[normalem]{ulem}
\usepackage{amsmath}
\usepackage{textcomp}
\usepackage{amssymb}
\usepackage{capt-of}
\usepackage{hyperref}
\author{Pan An}
\date{\textit{[2016-07-26 Tue]}}
\title{Remote access}
\hypersetup{
 pdfauthor={Pan An},
 pdftitle={Remote access},
 pdfkeywords={beamer org orgmode},
 pdfsubject={Example of using org to create presentations using the beamer exporter},
 pdfcreator={Emacs 24.5.1 (Org mode 8.3.4)}, 
 pdflang={English}}
\begin{document}

\maketitle
\tableofcontents


\section{Introduction}
\label{sec:orgheadline2}
\subsection{Overview}
\label{sec:orgheadline1}
This presentation provides an illustration of some of the capabilities of the \textbf{Beamer} export in \texttt{org} mode:

\begin{enumerate}
\item simple slides (this one),
\item slides with special blocks,
\item multi-column slides and
\item the use of \textbf{Babel} for literate programming.
\end{enumerate}

This file should be exported using \texttt{M-x org-export-dispatch} specifying \texttt{l} for \LaTeX{} and then \texttt{P}, for instance, to generate the PDF.

\section{Methodology}
\label{sec:orgheadline12}

\subsection{A simple slide}
\label{sec:orgheadline3}
This slide consists of some text with a number of bullet points:

\begin{itemize}
\item the first, very \textbf{important}, point!
\item the previous point shows the use of \textbf{bold} emphasis which is translated to a \texttt{\textbackslash{}alert\{\}} directive in \LaTeX.
\end{itemize}

The above list could be numbered or any other type of list and may include sub-lists.

\subsection{A more complex slide}
\label{sec:orgheadline5}
This slide illustrates the use of Beamer blocks.  The following text,
with its own headline, is displayed in a block:
\begin{enumerate}
\item Org mode increases productivity\hfill{}\textsc{B\_theorem}
\label{sec:orgheadline4}
\begin{itemize}
\item org mode means not having to remember \LaTeX commands.
\item it is based on ASCII text which is inherently portable.
\item Emacs!
\end{itemize}

\hfill \(\qed\)
\end{enumerate}

\subsection{Two columns}
\label{sec:orgheadline8}

\begin{enumerate}
\item A block\hfill{}\textsc{BMCOL}
\label{sec:orgheadline6}
\begin{itemize}
\item this slide consists of two columns
\item the first (left) column has no heading and consists of text
\item the second (right) column has an image and is enclosed in an \textbf{example} block
\end{itemize}

\item A screenshot\hfill{}\textsc{B\_example:BMCOL}
\label{sec:orgheadline7}
\includegraphics[width=.9\linewidth]{../../images/org-beamer/a-simple-slide.png}
\end{enumerate}

\subsection{Babel}
\label{sec:orgheadline11}
This slide shows some code and resulting output using \textbf{Babel}.  Note the specification of \texttt{BEAMER\_act} property for the second column.
\begin{enumerate}
\item Octave code\hfill{}\textsc{BMCOL:B\_block}
\label{sec:orgheadline9}
\begin{verbatim}
A = [1 2 ; 3 4]
b = [1; 1];
x = A\b
\end{verbatim}

\item The output\hfill{}\textsc{BMCOL:B\_block}
\label{sec:orgheadline10}
\begin{verbatim}
A =

   1   2
   3   4

x =

  -1
   1
\end{verbatim}
\end{enumerate}

\section{Conclusions}
\label{sec:orgheadline14}

\subsection{Summary}
\label{sec:orgheadline13}
\begin{itemize}
\item org is an incredible tool for time management
\item \textbf{but} it is also excellent for writing and for preparing presentations
\item Beamer is a very powerful \LaTeX{} package for presentations
\item the combination is unbeatable!
\end{itemize}
\end{document}
