\documentclass[dvips,12pt]{article}

% Any percent sign marks a comment to the end of the line

% Every latex document starts with a documentclass declaration like this
% The option dvips allows for graphics, 12pt is the font size, and article
%   is the style

\usepackage[pdftex]{graphicx}
\usepackage{url}

% These are additional packages for "pdflatex", graphics, and to include
% hyperlinks inside a document.

\setlength{\oddsidemargin}{0.25in}
\setlength{\textwidth}{6.5in}
\setlength{\topmargin}{0in}
\setlength{\textheight}{8.5in}

% These force using more of the margins that is the default style

\begin{document}

% Everything after this becomes content
% Replace the text between curly brackets with your own

\title{Fluid Material with Interaction}
\author{Pan An}
\date{\today}

% You can leave out "date" and it will be added automatically for today
% You can change the "\today" date to any text you like


\maketitle

% This command causes the title to be created in the document

\section{Introduction}

% An article style is separated into sections and subsections with 
%   markup such as this.  Use \section*{Principles} for unnumbered sections.

Material rendering has been considered very important in computer
graphics. Real world materials have vast varieties of physical
characteristics, thus makes pure computer simulation challenging and
exciting.

Successful models for rendering varies materials include translucent
materials, fluid, fabric, sand, etc. Andre et al. has provided an
excellent model for the simulation of porous sand and water
mixture. (Here put some other examples).

While recent development has made it possible for computers to
simulate real world materials with a rather high performance, the
process of rendering is considered to be heavily reliant on
computational powers. 

We propose a novel method to enhance user experience with digital
devices. Though current game engines has been proven to be very
powerful for development and deployment, there has been very few
topics on making digital and real images seamless at runtime. Our
proposed method provides an augmented reality interface for people to
interact with wet sand in virtual space. By providing prerendered
information for virtual devices, we will be able to greatly increase
CG effect on handheld devices, such as iPhone.  

\begin{figure}
\begin{center}
%\resizebox{6in}{!}{\includegraphics*{m42.jpg}}
\end{center}

\caption{AR/VR, Wet Sand Simulation}

\end{figure}


\section{Targets}

Our proposed method is made of two separated parts. 

\section{Filters, exposures, and special requests}

Assuming that the best telescope for your work is one of the two 0.5 meters
(CDK20N at Moore Observatory, CDK20S at Mt. Kent), you will have a choice of
filters:  Sloan filter set (g, r, i, or z),  Johnson-Cousins (U, B, V, R, or I),
color imaging (B, G, R, or clear), and narrow band (S $[II]$, red continuum,
H$\alpha$,  O$[III]$.  Identify which filters are of interest.

A typical exposure time for a magnitude 12 star to about half saturation is 100
seconds, but it depends on the filter choice.  Based on this, estimate how many
exposures you will need, and what total time you require.  In some cases, for
example studying an eclipsing or variable star, or an exoplanet transit, you
would use only one filter and make many measurements over a night.  In others,
you may make only a few exposures in each filter, and try many different
filters.   Changing filter sets takes an operator and several minutes, but
changing filters within one set (e.g. a different Sloan filter) takes only a few
seconds.

We have other telescopes that may be available at Moore Observatory this season.
There is a wide field astrograph that has a field of view of $4^\circ$ and is a
fast $f/4$,  especially good for large nebula, comets, or surveys.  A 14-inch
(0.36 meter) Celestron  telescope can be equipped with a fast camera for
planetary imaging.  A 27-inch (0.7 meter)  corrected Dall-Kirkham is scheduled
to be be delivered to Australia this fall, although we are unsure of the actual
date it could see light yet.  


 
\begin{thebibliography}{99}

\bibitem{gonzalez2012} Jonay I. Gonz\'{a}lez Hern\'{a}ndez, 
Pilar Ruiz-Lapuente,	
Hugo M. Tabernero,	
David Montes,	
Ramon Canal,	
Javier M\'{e}ndez	
and Luigi R. Bedin,
{No surviving evolved companions of the progenitor of SN1006},
Nature, {\bf 489}, 533-536 (2012).

\end{thebibliography}



\end{document}
%%% Local Variables:
%%% mode: latex
%%% TeX-master: t
%%% End:
