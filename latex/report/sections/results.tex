\chapter{Results }
\section{Fast Motion Segmentation}
We characterized our saccade based framework by experiments in
controlled environments. Figure 5 shows a performance map where each
point represents an experiment color coded with performance (see
sidebar). Parameters for saccades are shown alongside y-axis while
x-axis evaluates the algorithm’s ability to classify various
velocities of moving objects. Performance of motion segmentation was
best when micro-motion profile was used in the robot.

This Figure shows some sample segmentation results from the datasets
we created. Dataset-1 has data from an oscillating pendulum like
object was recorded at varying velocity transition levels. This
dataset was recorded with both high and low background clutter. Using
an oscillating object in the dataset allowed for a wide variety of
object velocities, from 0 to a maximum pendulum velocity of
1.28m/s. Dataset-2 was collected to quantify the effect of important
micro-motion parameters on classification accuracies. The background
was made up of black stripes (10cm each) on a white board, with a
horizontal and vertical separation of 15cm and 10.5cm,
respectively. The foreground object was a ball with a 5cm diameter
rotating on a 6cm rod. This work will be submitted to Frontiers in
Neuroscience in August 2016.

\section{Vibro-Tactile Haptic Glove}

%%% Local Variables:
%%% mode: latex
%%% TeX-master: "../master"
%%% End:
